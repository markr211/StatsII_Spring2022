\documentclass[12pt,letterpaper]{article}
\usepackage{graphicx,textcomp}
\usepackage{natbib}
\usepackage{setspace}
\usepackage{fullpage}
\usepackage{color}
\usepackage[reqno]{amsmath}
\usepackage{amsthm}
\usepackage{fancyvrb}
\usepackage{amssymb,enumerate}
\usepackage[all]{xy}
\usepackage{endnotes}
\usepackage{lscape}
\newtheorem{com}{Comment}
\usepackage{float}
\usepackage{hyperref}
\newtheorem{lem} {Lemma}
\newtheorem{prop}{Proposition}
\newtheorem{thm}{Theorem}
\newtheorem{defn}{Definition}
\newtheorem{cor}{Corollary}
\newtheorem{obs}{Observation}
\usepackage[compact]{titlesec}
\usepackage{dcolumn}
\usepackage{tikz}
\usetikzlibrary{arrows}
\usepackage{multirow}
\usepackage{xcolor}
\newcolumntype{.}{D{.}{.}{-1}}
\newcolumntype{d}[1]{D{.}{.}{#1}}
\definecolor{light-gray}{gray}{0.65}
\usepackage{url}
\usepackage{listings}
\usepackage{color}

\definecolor{codegreen}{rgb}{0,0.6,0}
\definecolor{codegray}{rgb}{0.5,0.5,0.5}
\definecolor{codepurple}{rgb}{0.58,0,0.82}
\definecolor{backcolour}{rgb}{0.95,0.95,0.92}

\lstdefinestyle{mystyle}{
	backgroundcolor=\color{backcolour},   
	commentstyle=\color{codegreen},
	keywordstyle=\color{magenta},
	numberstyle=\tiny\color{codegray},
	stringstyle=\color{codepurple},
	basicstyle=\footnotesize,
	breakatwhitespace=false,         
	breaklines=true,                 
	captionpos=b,                    
	keepspaces=true,                 
	numbers=left,                    
	numbersep=5pt,                  
	showspaces=false,                
	showstringspaces=false,
	showtabs=false,                  
	tabsize=2
}
\lstset{style=mystyle}
\newcommand{\Sref}[1]{Section~\ref{#1}}
\newtheorem{hyp}{Hypothesis}

\title{Problem Set 4}
\date{Due: April 4, 2022}
\author{Applied Stats II}


\begin{document}
	\maketitle
	\section*{Instructions}
	\begin{itemize}
		\item Please show your work! You may lose points by simply writing in the answer. If the problem requires you to execute commands in \texttt{R}, please include the code you used to get your answers. Please also include the \texttt{.R} file that contains your code. If you are not sure if work needs to be shown for a particular problem, please ask.
		\item Your homework should be submitted electronically on GitHub in \texttt{.pdf} form.
		\item This problem set is due before class on Monday April 4, 2022. No late assignments will be accepted.
		\item Total available points for this homework is 80.
	\end{itemize}

	\vspace{.25cm}
\section*{Question 1}
\vspace{.25cm}
\noindent We're interested in modeling the historical causes of infant mortality. We have data from 5641 first-born in seven Swedish parishes 1820-1895. Using the "infants" dataset in the \texttt{eha} library, fit a Cox Proportional Hazard model using mother's age and infant's gender as covariates. Present and interpret the output.

Firstly, install the libraries and load in the dataset (see Table 1; pg. 2). The cox proportional hazard model can be used from the survival package and the dataset here is the 'infants' data from the eha library. Then call the help file on this dataset to make assumptions about the outocome and predictor variables that will be used in this analysis.  Importantly, the 'sex' predictor is a dummy categorical variable and the 'age' predictor is a numeric variable. 

% Table created by stargazer v.5.2.2 by Marek Hlavac, Harvard University. E-mail: hlavac at fas.harvard.edu
% Date and time: Sun, Apr 03, 2022 - 20:03:13
\begin{table}[!htbp] \centering 
  \caption{} 
  \label{} 
\begin{tabular}{@{\extracolsep{5pt}} ccccccccccc} 
\\[-1.8ex]\hline 
\hline \\[-1.8ex] 
stratum & enter & exit & event & mother & age & sex & parish & civst & ses & year \\ 
\hline \\[-1.8ex] 
$1$ & $55$ & $365$ & $0$ & dead & $26$ & boy & Nedertornea & married & farmer & $1,877$ \\ 
$1$ & $55$ & $365$ & $0$ & alive & $26$ & boy & Nedertornea & married & farmer & $1,870$ \\ 
$1$ & $55$ & $365$ & $0$ & alive & $26$ & boy & Nedertornea & married & farmer & $1,882$ \\ 
$2$ & $13$ & $76$ & $1$ & dead & $23$ & girl & Nedertornea & married & other & $1,847$ \\ 
\hline \\[-1.8ex] 
\end{tabular} 
\end{table} 

	\lstinputlisting[language=R, firstline=12,lastline=24]{../my_answers/PS4_mark_roche.R} 

Now, create the survival rate variable using the enter, exit, event variables. Then, fit the cox proportional hazards model using the coxph() function. Summarize the output using the summary() function. Finally, interpret this output

	\lstinputlisting[language=R, firstline=26,lastline=31]{../my_answers/PS4_mark_roche.R} 
	
	\begin{verbatim}

			THE COX-PROPORTIONAL HAZARDS SUMMARY OUTPUT
Call:
coxph(formula = inf_surv ~ age + sex, data = data)

n= 105, number of events= 21 

				coef 	exp(coef) 	se(coef)      z 	Pr(>|z|)
age    -0.04044   0.96037  0.04507 -0.897    0.370
sexboy -0.48518   0.61559  0.44224 -1.097    0.273

			exp(coef)	exp(-coef)	lower .95 	upper .95
age       0.9604      1.041    		0.8792     		1.049
sexboy    0.6156     1.624    	0.2587     		1.465

Concordance= 0.586  (se = 0.058 )
Likelihood ratio test= 1.99  on 2 df,   p=0.4
Wald test            = 2  on 2 df,   p=0.4
Score (logrank) test = 2.03  on 2 df,   p=0.4
	\end{verbatim}

	\vspace{.25cm}
	
INTERPRETING THE OUTPUT 
	\vspace{.25cm}
	
The sample size of infants is 105. The no. of deaths is 21
	\vspace{.25cm}
	
 Coefficients:

Coefficient for sexboy is -0.04. The expected log of the hazard decreases by -0.04 when an infant is a boy compared to a girl, holding age constant.

Coefficient for age is -0.49. The expected log of the hazard decreases by -0.49 if a mother is one year older, holding the sex of the infant constant.
	\vspace{.25cm}
	
Exponate the coefficients for hazard ratios. This information is already given in the summary output.

Both hazard ratios are below 1 (i.e. they are associated with increased survivablity). Hazard ratio for sexboy (i.e. exp(coef)) = 0.62. Therefore, the hazard ratio for boys as opposed to girls is  62 percent; in other words, out of 100 infants, 62 that die are most likley to be male as to apposed to 58 females. 

Hazard ratio for age = 0.96. 
	\vspace{.25cm}
	
P-values: None of the coefficients are statistically significant

Run a chi-square test to test the null hypothesis that the hazards are proportional. 

\lstinputlisting[language=R, firstline=50,lastline=53]{../my_answers/PS4_mark_roche.R}
\end{document}
