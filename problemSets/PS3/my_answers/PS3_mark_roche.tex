\documentclass[12pt,letterpaper]{article}
\usepackage{graphicx,textcomp}
\usepackage{natbib}
\usepackage{setspace}
\usepackage{fullpage}
\usepackage{color}
\usepackage[reqno]{amsmath}
\usepackage{amsthm}
\usepackage{fancyvrb}
\usepackage{amssymb,enumerate}
\usepackage[all]{xy}
\usepackage{endnotes}
\usepackage{lscape}
\newtheorem{com}{Comment}
\usepackage{float}
\usepackage{hyperref}
\newtheorem{lem} {Lemma}
\newtheorem{prop}{Proposition}
\newtheorem{thm}{Theorem}
\newtheorem{defn}{Definition}
\newtheorem{cor}{Corollary}
\newtheorem{obs}{Observation}
\usepackage[compact]{titlesec}
\usepackage{dcolumn}
\usepackage{tikz}
\usetikzlibrary{arrows}
\usepackage{multirow}
\usepackage{xcolor}
\newcolumntype{.}{D{.}{.}{-1}}
\newcolumntype{d}[1]{D{.}{.}{#1}}
\definecolor{light-gray}{gray}{0.65}
\usepackage{url}
\usepackage{listings}
\usepackage{color}

\definecolor{codegreen}{rgb}{0,0.6,0}
\definecolor{codegray}{rgb}{0.5,0.5,0.5}
\definecolor{codepurple}{rgb}{0.58,0,0.82}
\definecolor{backcolour}{rgb}{0.95,0.95,0.92}

\lstdefinestyle{mystyle}{
	backgroundcolor=\color{backcolour},   
	commentstyle=\color{codegreen},
	keywordstyle=\color{magenta},
	numberstyle=\tiny\color{codegray},
	stringstyle=\color{codepurple},
	basicstyle=\footnotesize,
	breakatwhitespace=false,         
	breaklines=true,                 
	captionpos=b,                    
	keepspaces=true,                 
	numbers=left,                    
	numbersep=5pt,                  
	showspaces=false,                
	showstringspaces=false,
	showtabs=false,                  
	tabsize=2
}
\lstset{style=mystyle}
\newcommand{\Sref}[1]{Section~\ref{#1}}
\newtheorem{hyp}{Hypothesis}

\title{Problem Set 3}
\date{Due: March 28, 2022}
\author{Applied Stats II}


\begin{document}
	\maketitle
	\section*{Instructions}
	\begin{itemize}
		\item Please show your work! You may lose points by simply writing in the answer. If the problem requires you to execute commands in \texttt{R}, please include the code you used to get your answers. Please also include the \texttt{.R} file that contains your code. If you are not sure if work needs to be shown for a particular problem, please ask.
		\item Your homework should be submitted electronically on GitHub in \texttt{.pdf} form.
		\item This problem set is due before class on Monday March 28, 2022. No late assignments will be accepted.
		\item Total available points for this homework is 80.
	\end{itemize}

	\vspace{.25cm}
\section*{Question 1}
\vspace{.25cm}
\noindent We are interested in how governments' management of public resources impacts economic prosperity. Our data come from \href{https://www.researchgate.net/profile/Adam_Przeworski/publication/240357392_Classifying_Political_Regimes/links/0deec532194849aefa000000/Classifying-Political-Regimes.pdf}{Alvarez, Cheibub, Limongi, and Przeworski (1996)} and is labelled \texttt{gdpChange.csv} on GitHub. The dataset covers 135 countries observed between 1950 or the year of independence or the first year forwhich data on economic growth are available ("entry year"), and 1990 or the last year for which data on economic growth are available ("exit year"). The unit of analysis is a particular country during a particular year, for a total $>$ 3,500 observations. 

\begin{itemize}
	\item
	Response variable: 
	\begin{itemize}
		\item \texttt{GDPWdiff}: Difference in GDP between year $t$ and $t-1$. Possible categories include: "positive", "negative", or "no change"
	\end{itemize}
	\item
	Explanatory variables: 
	\begin{itemize}
		\item
		\texttt{REG}: 1=Democracy; 0=Non-Democracy
		\item
		\texttt{OIL}: 1=if the average ratio of fuel exports to total exports in 1984-86 exceeded 50\%; 0= otherwise
	\end{itemize}
	
\end{itemize}
\newpage
\noindent Please answer the following questions:

\begin{enumerate}
	\item Construct and interpret an unordered multinomial logit with \texttt{GDPWdiff} as the output and "no change" as the reference category, including the estimated cutoff points and coefficients.

Firstly, run descriptive statistics on the dataset using the summary() function. Then create a table consisting of the X variables OIL and REG to get a more immediate understanding on this data. As you can see below, the vast majority of fuel exports do not exceed 50 percent. By comparison, there is a similar sample of democratic and nn-democratic countries. 
	
		\lstinputlisting[language=R, firstline=24,lastline=36]{../my_answers/PS3_mark_roche.R} 
	
Now, manipulate the outcome variable to make relevant character categories out of the numeric values. Then, turn it into a factor and set 'no  change' as the reference catgeory. With this complete, create two dummy variables out of both the democratic and OIL predictors. After this, create a new smaller dataset containing the cleaned outcome and predictor variables. This dataset is then ready for unordered multinomil logit.  
 
 	\lstinputlisting[language=R, firstline=38,lastline=67]{../my_answers/PS3_mark_roche.R} 
 	
 		\begin{verbatim}
 	Table 1: The Unordered Multinomial Regression Output 
 	
 		Coefficients:
         (Intercept) large_oilex small_oilex democracy non_democracy
negative    5.145122    8.385181   -3.240059  3.248796      1.896326
positive    5.554894    8.486243   -2.931349  3.648636      1.906258

Std. Errors:
         (Intercept) large_oilex small_oilex democracy non_democracy
negative   0.1522334   0.0350539   0.1534428 0.4389436     0.3265688
positive   0.1519580   0.0350539   0.1529560 0.4380432     0.3257614

Residual Deviance: 4678.726 
AIC: 4690.726 
 	\end{verbatim}
 
 Now, interpret the coefficients and the intercepts. 
 
 	\begin{verbatim}
 1) The Coefficients
 
 A one-unit increase in the variable large_oilex is associated with a large increase in log odds of being negative difference between GDP rather than no change GDP difference in the amount of 8.385.

A one-unit increase in the variable large_oilex is associated with a large increase in log odds of being positive difference between GDP rather than no change GDP difference in the amount of 8.486.

A one-unit increase in the variable small_oilex is associated with a decrease in log odds of being negative difference between GDP rather than no change GDP difference in the amount of -3.240

A one-unit increase in the variable small_oilex is associated with a decrease in log odds of being positive difference between GDP rather than no change GDP difference in the amount of -2.931

A one-unit increase in the variable democracy is associated with an increase in log odds of being negative difference between GDP rather than no change GDP difference in the amount of 3.249
 
 A one-unit increase in the variable democracy is associated with an increase in log odds of being positive difference between GDP rather than no change GDP difference 
in the amount of 3.649
 
 A one-unit increase in the variable non_democracy is associated with a slight increase in log odds of being negative difference between GDP rather than no change GDP difference in the amount of 0.327
 
 A one-unit increase in the variable non_democracy is associated with a slight increase in log odds of being positive difference between GDP rather than no change GDP difference in the amount of 0.326
 	
 2) The Intercepts:
 
 The intercept for negative GDP diff: when all predictors are at 0, the log odds of there being negative difference between GDP rather than no change GDP difference is associated with an increase in the amount of  5.554894
 
 The intercept for positive GDP diff: when all predictors are at 0, the log odds of there being positive difference between GDP rather than no change GDP difference is associated with an increase in the amount of 5.145122
 \end{verbatim}

Now, run a z-test and get a p-value to look for significance. Drop any predictor from model that is not signficant. Call this model the final model. Then run a final unordered multinomial regression and reinterpret the output.

	\lstinputlisting[language=R, firstline=117,lastline=159]{../my_answers/PS3_mark_roche.R} 
	
	\item Construct and interpret an ordered multinomial logit with \texttt{GDPWdiff} as the outcome variable, including the estimated cutoff points and coefficients.
	
Order the outcome variable GDPWdiff as an ordered factor by levels ranging from positive to no change, to negative. Then, run an ordered logit regression using the polyr() function and summarise the output. 

	\lstinputlisting[language=R, firstline=166,lastline=173]{../my_answers/PS3_mark_roche.R} 
	 
	 	\begin{verbatim}
	              Ordinal Regression Output 
	Call:
	polr(formula = GDPWdiff ~ ., data = final_data, Hess = TRUE)
	
	Coefficients:
	Value   Std. Error t value
	OIL           -0.1788    0.11546  -1.549
	REG           0.4102    0.07518   5.456
	
	Intercepts:
										Value    Std. Error 	t value 
	no change|negative  -5.3199   0.2523   -21.0865
	negative|positive   -0.7036   0.0476   -14.7932
	
	Residual Deviance: 4686.606 
	AIC: 4694.606 
	\end{verbatim}

The output of the model is visualised above. The coefficient for the OIL variable is -0.1788 and the coefficient for the REG variable is 0.4102. Next, a significance test can be run on each coefficient to validate these values via a p value. This is shown below. 

	\lstinputlisting[language=R, firstline=192,lastline=199]{../my_answers/PS3_mark_roche.R} 

Therefore, both coefficients appear significant. Now we can interpret these coefficients in English by exponating them to get the odds ratio. This is shown in code below.

	\lstinputlisting[language=R, firstline=200,lastline=211]{../my_answers/PS3_mark_roche.R} 

Therefore,  parts 1 and 2 to this Problem Set have run and interpreted the cutoff points and coefficients for an unordered multinomial and an ordered regression model. 

\end{enumerate}

\section*{Question 2} 
\vspace{.25cm}

\noindent Consider the data set \texttt{MexicoMuniData.csv}, which includes municipal-level information from Mexico. The outcome of interest is the number of times the winning PAN presidential candidate in 2006 (\texttt{PAN.visits.06}) visited a district leading up to the 2009 federal elections, which is a count. Our main predictor of interest is whether the district was highly contested, or whether it was not (the PAN or their opponents have electoral security) in the previous federal elections during 2000 (\texttt{competitive.district}), which is binary (1=close/swing district, 0="safe seat"). We also include \texttt{marginality.06} (a measure of poverty) and \texttt{PAN.governor.06} (a dummy for whether the state has a PAN-affiliated governor) as additional control variables. 

\begin{enumerate}
	\item [(a)]
	Run a Poisson regression because the outcome is a count variable. Is there evidence that PAN presidential candidates visit swing districts more? Provide a test statistic and p-value.

	\item [(b)]
	Interpret the \texttt{marginality.06} and \texttt{PAN.governor.06} coefficients.
	
	\item [(c)]
	Provide the estimated mean number of visits from the winning PAN presidential candidate for a hypothetical district that was competitive (\texttt{competitive.district}=1), had an average poverty level (\texttt{marginality.06} = 0), and a PAN governor (\texttt{PAN.governor.06}=1).
	
\end{enumerate}

\end{document}
